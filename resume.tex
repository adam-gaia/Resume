% Adam Gaia
% Resume

\documentclass{article}
\usepackage{xcolor}
\usepackage{xparse}
\usepackage{multicol}
\usepackage{enumitem}

% Margin setup. Modified from Scott Clark's margin setup on his own resume. https://github.com/sc932/resume
\setlength{\evensidemargin}{-0.25in}
\setlength{\headheight}{-0.25in}
\setlength{\headsep}{0in}
\setlength{\oddsidemargin}{-0.25in}
\setlength{\paperheight}{11in}
\setlength{\paperwidth}{8.5in}
\setlength{\textheight}{9.75in}
\setlength{\textwidth}{7in}
\setlength{\topmargin}{-0.3in}
\setlength{\topskip}{0in}
\setlength{\voffset}{0.1in}


\setlist{nolistsep} % Turn off whitespace between bullet points
\pagenumbering{gobble} % Turn off page numbering
\definecolor{titlebox}{gray}{0.70}  % Inner background color of title bar
\setlength{\multicolsep}{0.0pt plus 0.0pt minus 0.0pt} % Fix spacing before and after multicols

% Command to create a section separation bar. Inspired by Scott Clark's separation bars on his own resume.  https://github.com/sc932/resume
\NewDocumentCommand{\sepperator}{mmmm}
 {
    % #1 = width
    % #2 = frame color
    % #3 = background color
    % #4 = text
    \begin{center}
    \fcolorbox{#2}{#3}{\makebox[#1]{\large{\textbf{#4}}}}
    \end{center}
    % TODO: Different font
 }

\begin{document}


% --------------------------------------------------
% Contact Info
% --------------------------------------------------
\begin{center}
{\huge\textbf{Adam Gaia}}
\end{center}

\begin{center}
\begin{tabular}{ c c c }
% The Make file will determine what contact info gets included by writing to ./contactInfo.txt
\input{./contactInfo.txt}
\end{tabular}
\end{center}


% --------------------------------------------------
% Work Experience
% --------------------------------------------------
\sepperator{7in}{titlebox}{titlebox!40}{Work Experience}
\textbf{Software Engineer - Sarcos Robotics} \hfill  November 2019 (intern) to present (full time)
\begin{itemize}
    \item Robot Service Manager
	\begin{itemize}
		\item Built and maintained an on-robot daemon that provided a REST API for interacting with services running on the robots

		\item Reduced the firmware update process from 1 hour to 10 minutes by building a CD pipeline distribute firmware to robots without human interaction

		\item Worked closely with test engineers to create internal tools for starting/stopping robot operation

		\item Used websockets to stream filtered runtime logs to robot deploy stations during robot operation

		\item Created and containerized on-robot services as dockerized python apps

		\item Wrote/tested/debugged python code
	\end{itemize}
	\vspace{1mm}

	\item Reduced the on-robot team's main codebase CI runtime from 40 minutes to 5 minutes by removing redundancy and running jobs in parallel

	\item Created a pipeline to transfer 100s of GB runtime logs from the robot to cloud storage after operation

	\item Worked with test engineers to debug and resolve robot bring-up issues on the test floor

	\item Wrote Ansible scripts to provision robot deploy stations (linux computers)

    %\item Lead developer of an on-robot, service-queuing daemon. Created and containerized services using Docker

    %\item Wrote/optimized/debugged c++ code for realtime robot operating system

    %\item Automated robot bring-up, software update, and code compile processes \\
\end{itemize}


\vspace{2mm}

% Uintah Scientific Computing
\noindent \textbf{Scientific Computing Intern - University of Utah} \hfill  August 2017 to November 2019
%\textit{University of Utah Engineering Department, Uintah Project (PDE-solving simulation software)}
\begin{itemize}
    \item Parallelized a post-processing script, reducing run-time from 18 hours to 3 minutes with 400+ GB input files
    \item Created Bash scripts to automate the queuing of remote simulations. Added automatic job error feedback
    %\item Created a Python script to recursively compare tree-structured simulation spec files for similar nodes
    \item Used Linux command line to run simulations on remote high-performance computing centers
    %\item Saved countless hours of time by writing Bash scripts to automate repetitive tasks
    %\item Updated legacy MATLAB post-processing tools to handle current outputs
    %\item Documented 3 new features in the user guide.
\end{itemize}


% --------------------------------------------------
% Computer Science and Engineering Projects
% --------------------------------------------------
\sepperator{7in}{titlebox}{titlebox!40}{Computer Science and Engineering Projects}
\begin{itemize}
    \item \textbf{4-Node Raspberry PI Server:}
    \textit{Headless home server built by linking 4 raspberry pi (linux) computers}
    \begin{itemize}
        \item Wrote Ansible scripts to provision nodes and automate server administration tasks
        \item Used docker to run containerized applications on the server
    \end{itemize}
\end{itemize}
\vspace{2mm}
\begin{itemize}
    \item \textbf{Spreadsheet Application:}
    \textit{Semester-long project to build an application from scratch in C\#}
    \begin{itemize}
        \item Used modular programming and MVC to combine individual components into a fully developed application
        \item Used hash maps to keep track of cell dependencies to optimize formula calculation speed
        \item Received an A on the project
    \end{itemize}
\vspace{2mm}
    \item \textbf{Ping-Pong Ball Launcher:}
        \textit{Project goal was to hit targets ranging between .5-1 meter away}
        \begin{itemize}
            \item Microcontroller set firing velocity, launcher position, and launch angle
            \item Processed an overhead image to find target location
            \item 8th place in timed competition (out of 100)
        \end{itemize}
\end{itemize}


% --------------------------------------------------
% Education
% --------------------------------------------------
\sepperator{7in}{titlebox}{titlebox!40}{Education}

\noindent \textbf{B.S. Computer Science} - 3.2 GPA \hfill on hold \\
\textit{University of Utah}
\begin{itemize}
\begin{multicols}{2}
    \item Object-Oriented Programming
    \item Software Practice 1 and 2
    \item Algorithms and Data Structures
    \item Engineering Probability and Statistics
    \item Professional Communication for Engineers
    \item Scientific Computing
    %\item Numerical Methods for Engineering Systems
    \item Computer Organization and Architecture
    %\item Intro to Electrical Engineering
    \item Calculus 1, 2, and 3
    \item Linear Algebra and Differential Equations
    \item Advanced Programming for Comp. Design Problems
\end{multicols}
\end{itemize}


% --------------------------------------------------
% Skills
% --------------------------------------------------
\sepperator{7in}{titlebox}{titlebox!40}{Skills}
\begin{itemize}
    \item \textbf{Primary Languages:} Python, Rust, Bash
    \item \textbf{Secondary Languages:} C++, C\#, Java
    \item \textbf{Tools:} Git, unit tests, integration tests, CI/CD, Docker, Unix/Linux, code review, \LaTeX %Arduino microcontrollers
    \item \textbf{Soft Skills:}  Outgoing team player, time management, oral and written communication skills
\end{itemize}


% --------------------------------------------------
% Old Techinical Courses Section - now moved to education section

% --------------------------------------------------
% Technical Courses
% --------------------------------------------------
%\sepperator{7in}{titlebox}{titlebox!40}{Technical Courses Taken}
%\noindent Technical Courses Taken:
% \begin{itemize}
% \begin{multicols}{2}
%     \item <class1>
%     \item <class2>
%     \item ...
% \end{multicols}
% \end{itemize}




\end{document}
